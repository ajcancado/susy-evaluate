\documentclass[12pt,a4paper]{report}
\usepackage[brazil]{babel}

\usepackage[style=numeric,backend=biber]{biblatex}
\usepackage[utf8]{inputenc}
\usepackage{kpfonts}
\usepackage[T1]{fontenc}
% \usepackage{lineno,hyperref}
\usepackage{wrapfig}
\usepackage{graphicx}
\usepackage{enumerate}
\usepackage{subcaption}
\usepackage{float}
\usepackage{caption}
\usepackage{listings}
\usepackage{lipsum}
\usepackage{amsthm}
\usepackage{amssymb}
\usepackage{bm}
\usepackage{color}
\usepackage{afterpage}
\usepackage[inline]{enumitem}
\usepackage{pdflscape}
\usepackage{listingsutf8}
\usepackage{siunitx}

\lstset{frame=tb,
  aboveskip=2mm,
  belowskip=2mm,
  showstringspaces=false,
  columns=flexible,
  basicstyle=\footnotesize,,
  numbers=left,
  numbersep=5pt,
  stepnumber=1,
  breaklines=true,
  keepspaces=true,
  breakatwhitespace=true,
  showtabs=false,  
  tabsize=2
}

\definecolor{mGreen}{rgb}{0,0.6,0}
\definecolor{mGray}{rgb}{0.5,0.5,0.5}
\definecolor{mPurple}{rgb}{0.58,0,0.82}
\definecolor{backgroundColour}{rgb}{0.97,0.97,0.97}

\lstdefinestyle{CStyle}{
    backgroundcolor=\color{backgroundColour},   
    commentstyle=\color{mGreen},
    keywordstyle=\color{magenta},
    numberstyle=\tiny\color{mGray},
    stringstyle=\color{mPurple},
    basicstyle=\footnotesize,
    breakatwhitespace=false,         
    breaklines=true,                 
    captionpos=b,                    
    keepspaces=true,                 
    numbers=left,                    
    numbersep=5pt,                  
    showspaces=false,                
    showstringspaces=false,
    showtabs=false,                  
    tabsize=2,
    language=C
}


\pagenumbering{arabic}
\renewcommand{\thesection}{\arabic{section}}

\bibliography{ref}
\renewcommand{\contentsname}{Sumário}{\thispagestyle{empty}}
\renewcommand{\baselinestretch}{1.5} 

\begin{document}

\begin{titlepage}
    \begin{center}
        \vspace*{1cm}
        \includegraphics[width=0.25\textwidth]{Logo}\\
        \vspace{1.5cm}
        \Huge
    	\textbf{Documentação das Mensagens de Erro Reportadas pelo SuSy-avalia}\\
        \vspace{1.5cm}
        \Large
        \textbf{Equipe}: Arthur J. Cançado, Felipe R. Jensen, Gabriell Orisaka,   
 		Juliana L. G. Corá, Silvana Trindade e Yusseli L. M. Mendoza\\
        \vspace{1.2cm}
    	\Large 
    	Instituto de Computação\\
    % 	\vspace{0.4cm}
    	Universidade Estadual de Campinas\\
    	\vspace{1.5cm}
        Campinas, Junho de 2017.
    \end{center}
\end{titlepage}
\tableofcontents
\clearpage

\section{Introdução}
Ferramenta para avaliação de código escrito em linguagem C e C++, para ser integrado ao sistema SuSy.
A avaliação feita pelo software é conhecida por análise estática, e é utilizada para dar \textit{feedback} ao desenvolvedor sobre possíveis problemas em seu código, tais como vazamento de memória, acesso à posições inválidas de ponteiros e vetores, etc. 
Este \textit{feedback} é importante para qualquer nível de programador, desde iniciantes até programadores experientes dado à vasta cobertura da análise.

\section{Interpretação das Saídas}\label{interpretacao}
Nesta seção serão apresentadas exemplos arquivos $.c$ e saídas de erros geradas a partir deles.
Além disso, nos exemplos apresentados neste documento apresentam apenas os trechos de código onde os erros ocorrem, não sendo portanto programas completos.

\subsection{Variáveis não inicializadas}
As Variáveis não inicializadas podem gerar resultados indesejados, pois a linguagem C não faz inicialização automática das variáveis, sendo assim, o que tem inicialmente é 'lixo'.
Já as variáveis não utilizadas geram o desperdício do uso dos recursos.
\subsubsection{Variável não inicializada}
Mensagem de erro: '[exemplo.c:3]: (erro) Variável não inicializada'.
\begin{lstlisting}[style=CStyle]
void f() {
    int a;
    a++;
}
\end{lstlisting}
Pontos a serem percebidos:
\begin{itemize}
    \item \textbf{exemplo.c} representa ao arquivo onde o erro foi encontrado e \textbf{3} representa a linha que o erro ocorreu.
    \item Na sequência é descrito o que motivou o erro a ocorrer, neste caso \textbf{Variável não inicializada}.
\end{itemize}


\subsubsection{Memória foi alocada mas não foi inicializada}
Mensagem de erro: '[exemplo.c:4]: (erro) Memória foi alocada mas não foi inicializada'.
\begin{lstlisting}[style=CStyle]
void f(){
   char *strMsg = "This is a message";
   char *buffer=(char*) malloc (128*sizeof(char));
   strcpy(strMsg,buffer);
   free(buffer);
}
\end{lstlisting}

\subsubsection{Variável foi criada em uma \textit{struct} mas não foi inicializada}
Mensagem de erro: '[exemplo.c:2]: (erro) Variável foi criada em uma \textit{struct} mas não foi inicializada'.
\begin{lstlisting}[style=CStyle]
struct AB { int a; int b; };
void f(void) {
    struct AB ab;
    int a = ab.a;
}
\end{lstlisting}

\subsection{Função e Variável não Utilizada}
Variáveis e funções criadas mas nunca utilizadas demandam recursos e impactam na hora da execução, gerando assim o desperdícios dos recursos utilizados.
\subsubsection{Função não utilizada}
Mensagem de erro: '[exemplo.c:1]: (erro) Função criada mas nunca foi utilizada'.
\begin{lstlisting}[style=CStyle]
int f() { printf("Hello World!"); }
\end{lstlisting}

\subsubsection{Variável nunca foi utilizada}
Mensagem de erro: '[exemplo.c:8]: (erro) Variável nunca foi utilizada'.
\begin{lstlisting}[style=CStyle]
void foo(int x) {
    int i;
    if (x) {
        int i;
    } else {
        int i;
    }
    i = 0;
}
\end{lstlisting}

\subsubsection{Recursos alocados mas nunca utilizados}
Mensagem de erro: '[exemplo.c:2]: (erro) Recursos de memória alocados nunca foram utilizados'.
\begin{lstlisting}[style=CStyle]
void foo() {
    void* ptr = malloc(16);
    free(ptr);
}
\end{lstlisting}

\subsubsection{Atribuição de valor à variável mas nunca utilizado}
Mensagem de erro: '[exemplo.c:2]: (erro) Atribuição de valor à variável mas nunca utilizado'.
\begin{lstlisting}[style=CStyle]
void foo() {
    int i = 0;
}
\end{lstlisting}

\subsubsection{Variável foi criada em uma \textit{struct} mas não foi utilizada}
Mensagem de erro: '[exemplo.c:2]: (erro) Variável foi criada em uma \textit{struct} mas não foi utilizada'.
\begin{lstlisting}[style=CStyle]
struct Point
{
    int x;
};
\end{lstlisting}

\subsubsection{Variável não foi atribuída}
Mensagem de erro: '[exemplo.c:2]: (erro) Variável não foi atribuída'.
\begin{lstlisting}[style=CStyle]
int foo() {
    int i;
    return i;
}
\end{lstlisting}

\subsubsection{Label não utilizado}
Mensagem de erro: '[exemplo.c:3]: (erro) Label não utilizado'.
\begin{lstlisting}[style=CStyle]
int f(char art) {
    switch (art) {
        caseZERO:
            return 0;
        case1:
            return 1;
        case 2:
            return 2;
    }
}
//...
\end{lstlisting}

\subsection{Verificação de Tipo de Dados}
Quando o tipo do dado está incorreto, consequentemente o resultado final será incorreto ou indefinido, como por exemplo $\log{0}$ que é indefinido.

\subsubsection{Resultado indefinido}
Mensagem de erro: '[exemplo.c:3]: (erro) Passando valor que pode gerar resultado indefinido'.
\begin{lstlisting}[style=CStyle]
void foo() {
    print(exp(x) - 1);
    print(log(1 + x));
    print(1 - erf(x));
}
\end{lstlisting}

\subsubsection{Resultado inteiro é atribuído a valor longo}
Mensagem de erro: '[exemplo.c:2]: (erro) Resultado inteiro é atribuído a valor longo, podendo ocasionalmente haver perda de informação'.
\begin{lstlisting}[style=CStyle]
long f(int x, int y) {
    const long ret = x * y;
    return ret;
}
\end{lstlisting}

\subsubsection{Resultado inteiro é retornado como valor longo}
Mensagem de erro: '[exemplo.c:2]: (erro) Resultado inteiro é retornado como valor longo, podendo ocasionalmente haver perda de informação'.
\begin{lstlisting}[style=CStyle]
long f(int x, int y) {
    return x * y;
}
\end{lstlisting}

\subsubsection{Overflow de variável do tipo \textit{float}}
Mensagem de erro: '[exemplo.c:2]: (erro) Overflow de variável do tipo \textit{float}'.
\begin{lstlisting}[style=CStyle]
void f(void) {
    return (int)1E100;
}
\end{lstlisting}

\subsubsection{Valor booleano atribuído a ponteiro}
Mensagem de erro: '[exemplo.c:2]: (erro) Valor booleano atribuído a ponteiro'.
\begin{lstlisting}[style=CStyle]
void f(char *p) {
    if (p+1){}
}
\end{lstlisting}

\subsubsection{Conversão literal de variável para booleano sempre é verdadeiro}
Mensagem de erro: '[exemplo.c:2]: (erro) Conversão literal de variável tipo caractere para booleano sempre é verdadeiro'.
\begin{lstlisting}[style=CStyle]
int f() {
    if ("Hello") { }
}
\end{lstlisting}

\subsubsection{Vetor acessado em índice inválido}
Mensagem de erro: '[exemplo.c:4]: (erro) Vetor acessado em índice inválido, portanto fora do seu limite'.
\begin{lstlisting}[style=CStyle]
int buffer[9];
long int i;
for(i = 10; i--;) {
    buffer[i] = i;
}
\end{lstlisting}

\subsubsection{Acesso inválido}
Mensagem de erro: '[exemplo.c:3]: (erro) Acesso inválido'.
\begin{lstlisting}[style=CStyle]
void f(unsigned char n) {
    int a[n];
    a[-1] = 0;
    a[256] = 0;
}
\end{lstlisting}

\subsubsection{Acesso a posição inválida do ponteiro}
Mensagem de erro: '[exemplo.c:4]: (erro) Acesso à posição inválida do ponteiro'.
\begin{lstlisting}[style=CStyle]
void f() {
    char a[10];
    char *p = a + 100;
}
\end{lstlisting}

\subsubsection{Possível acesso inválido}
Mensagem de erro: '[exemplo.c:3]: (erro) Possível acesso inválido'.
\begin{lstlisting}[style=CStyle]
void foo() {
    char *data = alloca(50);

    char src[100];
    memset(src,'C',99);
    src[99] = '\0';
    strcpy(data,src);
}
\end{lstlisting}
Este tipo de mensagem pode ser caracterizada como falso positivo, ou seja, o aluno deve verificar se realmente este erro ocorre.

\subsection{Sintaxe Suspeita}
Os erros reportados de sintaxe suspeita podem gerar resultados incorretos, entretanto o que deve ser analisado é se realmente o que é gerado um resultado incorreto, sendo assim, cabe ao aluno executar o programa e verificar se de fato a saída gerada sofre impacto pelos erros reportados.


\subsubsection{Atribuição redundante}
Mensagem de erro: '[exemplo.c:4]: (erro) Atribuição redundante'.
\begin{lstlisting}[style=CStyle]
void f() {
    int i[10];
    i[2] = 1;
    i[2] = 1;
}
\end{lstlisting}


\subsubsection{Condição indefinida}
Mensagem de erro: '[exemplo.c:2]: (erro) Condição não está claramente definida'.
\begin{lstlisting}[style=CStyle]
int f() {
   if (a & b == c){}
}
\end{lstlisting}

\subsubsection{Comparação de tipo booleano com tipo inteiro}
Mensagem de erro: '[exemplo.c:2]: (erro) Comparação de tipo booleano com tipo inteiro'.
\begin{lstlisting}[style=CStyle]
void f(int x, bool y) { if ( x != y ) {} }
\end{lstlisting}

\subsubsection{Condição sempre verdadeira ou falsa}

Mensagem de erro: '[exemplo.c:2]: (erro) Condição sempre verdadeira ou falsa'.
\begin{lstlisting}[style=CStyle]
int x = 123;
if (sizeof(char) != x) {}
\end{lstlisting}

\subsubsection{Divisão por zero}
Mensagem de erro: '[exemplo.c:4]: (erro) Divisão por zero'.
\begin{lstlisting}[style=CStyle]
int a = 2;
int b = 2;
//...
b = a / (a-b);
\end{lstlisting}

\subsection{Formatação de Caractere}
A formatação incorreta das saídas ou entradas do programa pode gerar a busca por erros não existentes no código por resultar em saídas ou entradas no formato incorreto.
\subsubsection{Número de argumentos inválidos - \textit{printf}}
Mensagem de erro: '[exemplo.c:4]: (erro) Número de variáveis utilizadas não corresponde com o número de formatadores'.
\begin{lstlisting}[style=CStyle]
void foo() {
    scanf("%5s", bar);
    scanf("%5[^~]", bar);
    scanf("aa%%s", bar);
    scanf("aa%d", &a);
    scanf("aa%ld", &a);
    scanf("%*[^~]");
}
\end{lstlisting}

\subsubsection{Scanf com argumento do tipo \textit{string}/\textit{char}* inválido}
Mensagem de erro: '[exemplo.c:3]: (erro) Scanf com argumento do tipo \textit{string}/\textit{char}* inválido'.
\begin{lstlisting}[style=CStyle]
void g() {
    const char c[]="42";
    scanf("%s", c);
}
\end{lstlisting}

\subsubsection{Scanf com argumento do tipo \textit{int} inválido}
Mensagem de erro: '[exemplo.c:5]: (erro) Scanf com argumento do tipo \textit{int} inválido'.
\begin{lstlisting}[style=CStyle]
void foo() {
    int i;
    unsigned int u;
    char str[10];
    scanf("%d", str);
}
\end{lstlisting}

\subsubsection{Scanf com argumento do tipo \textit{float} inválido}
Mensagem de erro: '[exemplo.c:5]: (erro) Scanf com argumento do tipo \textit{float} inválido'.
\begin{lstlisting}[style=CStyle]
void foo() {
    int i;
    unsigned int u;
    char str[10];
    scanf("%f", str);
}
\end{lstlisting}

\subsubsection{Argumento do tipo \textit{string}/\textit{char}* no formato apresentado é inválido}
Mensagem de erro: '[exemplo.c:3]: (erro) Argumento do tipo \textit{string}/\textit{char} no formato apresentado é inválido'.
\begin{lstlisting}[style=CStyle]
void foo(char* s, const char* s2, std::string s3, int i) {
    printf("%s%s", s, s2);
    printf("%s", i);
    printf("%i%s", i, i);
    printf("%s", s3);
    printf("%s", "s4");
    printf("%u", s);
}
\end{lstlisting}

\subsubsection{Argumento do tipo \textit{number} no formato apresentado é inválido}
Mensagem de erro: '[exemplo.c:2]: (erro) Argumento do tipo \textit{number} no formato apresentado é inválido'.
\begin{lstlisting}[style=CStyle]
void foo(const int ci) {
    printf("%n", ci);
}
\end{lstlisting}

\subsubsection{Argumento do tipo ponteiro no formato apresentado é inválido}
Mensagem de erro: '[exemplo.c:3]: (erro) Argumento do tipo ponteiro no formato apresentado é inválido'.
\begin{lstlisting}[style=CStyle]
double f() { return 0; }
void foo() { 
    printf("%p", f());
}
\end{lstlisting}

\subsubsection{Argumento de tipo \textit{int} é inválido neste caso}
Mensagem de erro: '[exemplo.c:3]: (erro) Argumento de tipo \textit{int} é inválido neste caso'.
\begin{lstlisting}[style=CStyle]
double f() { return 0; }
void foo() { 
    printf("%f %ol", f(), f());
}
\end{lstlisting}

\subsubsection{Argumento de tipo \textit{unsigned int} é inválido neste caso}
Mensagem de erro: '[exemplo.c:3]: (erro) Argumento de tipo \textit{unsigned int} é inválido neste caso'.
\begin{lstlisting}[style=CStyle]
class foo {};
void foo(const int* cpi, foo f, bar b, bar* bp, double d, int i, bool bo) {
    printf("%u", f);
    printf("%u", "s4");
    printf("%u", d);
    printf("%u", i);
    printf("%u", cpi);
    printf("%u", b);
    printf("%u", bp);
    printf("%u", bo);
}
\end{lstlisting}

\subsubsection{Argumento de tipo \textit{short int} inválido}
Mensagem de erro: '[exemplo.c:2]: (erro) Argumento de tipo \textit{short int} é inválido neste caso'.
\begin{lstlisting}[style=CStyle]
double f() { return 0; }
void foo() { 
    printf("%f %d", f(), f());
}
\end{lstlisting}

\subsubsection{Argumento de tipo \textit{float} inválido}
Mensagem de erro: '[exemplo.c:3]: (erro) Argumento de tipo \textit{float} é inválido neste caso'.
\begin{lstlisting}[style=CStyle]
class foo {};
    //...
    printf("%f", (float)cpi);
}
\end{lstlisting}

\subsubsection{Formato de \textit{string}/\textit{char}* modificado não pode ser utilizado}
Mensagem de erro: '[exemplo.c:2]: (erro) Formato de \textit{string}/\textit{char} modificado não pode ser utilizado sem conversão específica'.
\begin{lstlisting}[style=CStyle]
void foo(unsigned int i) {
    size_t s;
    printf("%I64%i", s, i);
}
\end{lstlisting}

\subsubsection{Tamanho da variável no formato \textit{string}/\textit{char}* é inválida}
Mensagem de erro: '[exemplo.c:2]: (erro) Tamanho da variável no formato \textit{string}/\textit{char} é inválida'.
\begin{lstlisting}[style=CStyle]
void f()
{
  char str [8];
  scanf ("%70s",str);
}
\end{lstlisting}

\subsubsection{Referenciamento de parâmetro foi realizado incorretamente}
Mensagem de erro: '[exemplo.c:3]: (erro) Referenciamento de parâmetro foi realizado incorretamente'.
\begin{lstlisting}[style=CStyle]
void foo() {
    int bar;
    scanf("%2$d", &bar);
    printf("%0$f", 0.0);
}
\end{lstlisting}


\subsection{Vazamento de Memória}
Os erros apontados pelo vazamento de memória estão relacionados a alocação e desalocação que podem gerar resultados indesejados ou ambiguidade no código.

\subsubsection{Vazamento de recursos}
Mensagem de erro: '[exemplo.c:4]: (erro) Vazamento de recursos'.
\begin{lstlisting}[style=CStyle]
void f() {
  FILE *fp = fopen("name", "r");
  if (!fp) {
    fp = fopen("name", "w");
    fclose(fp);
  }
}
\end{lstlisting}


\subsubsection{Vazamento de memória}
Mensagem de erro: '[exemplo.c:3]: (erro) Vazamento de memória'.
\begin{lstlisting}[style=CStyle]
void f() {
  int *p = malloc (sizeof(int)*10);
  return 1;
}
\end{lstlisting}

\subsubsection{Ponteiro liberado duas vezes}
Mensagem de erro: '[exemplo.c:4]: (erro) Ponteiro foi liberado duas vezes'.
\begin{lstlisting}[style=CStyle]
void foo()
{
    char *str = malloc(100);
    free(str);
    free(str);
}
\end{lstlisting}

\subsubsection{Acesso à variável já desalocada}
Mensagem de erro: '[exemplo.c:2]: (erro) Acesso à variável já desalocada'.
\begin{lstlisting}[style=CStyle]
void f(char *p) {
    free(p);
    *p = 0;
}
\end{lstlisting}

\subsubsection{Falta de alocação ou desalocação de memória}
Mensagem de erro: '[exemplo.c:2]: (erro) Falta de alocação ou desalocação de memória'.
\begin{lstlisting}[style=CStyle]
void f() {
    FILE*f=fopen(fname,a);
    free(f);
}
\end{lstlisting}

\subsubsection{Vazamento de memória no processo de realocação}
Mensagem de erro: '[exemplo.c:6]: (erro) Vazamento de memória no processo de realocação'.
\begin{lstlisting}[style=CStyle]
static void f() {
    char *buf = malloc(10);
    if (aa)
        // ...
    else if (buf = realloc(buf, 100))
       // ...
    free(buf);
}
\end{lstlisting}

\subsubsection{Liberação Dupla de Memória}
Mensagem de erro: '[exemplo.c:6]: (erro) Ponteiro foi liberado duas vezes'.
\begin{lstlisting}[style=CStyle]
void f(char *p) {
  if (x) {
    free(p);
    printf("Hello");
  }
  free(p);
}
\end{lstlisting}

\subsubsection{Retorno/acesso de variável já desalocada}
Mensagem de erro: '[exemplo.c:3]: (erro) Retorno/acesso de variável já desalocada'.
\begin{lstlisting}[style=CStyle]
void f(char *p) {
    free(p);
    return p;
}
\end{lstlisting}

\subsection{Ponteiro Nulo}
Verificação de ocorrência de erros na manipulação de ponteiros irá auxiliar em relação ao acesso de ponteiros que são nulos, e neste o SuSy-avalia faz a cobertura de três erros possíveis, que são os seguintes: aritmética de ponteiros e acesso a ponteiro nulo.

\subsubsection{Possível acesso a ponteiro nulo}
Mensagem de erro: '[exemplo.c:2]: (erro) Possível acesso a ponteiro nulo se o valor padrão foi utilizado'.
\begin{lstlisting}[style=CStyle]
void f(int *p = 0) {
    *p = 0;
}
\end{lstlisting}
Este erro só acontece em linguagem de programação C++.

\subsubsection{Possível valor de ponteiro ser nulo}
Mensagem de erro: '[exemplo.c:3]: (erro) Existe uma possibilidade do valor acessado do ponteiro ser nulo'.
\begin{lstlisting}[style=CStyle]
void foo(char *p) {
    //...
    if (!p) { }
}
\end{lstlisting}


\subsubsection{Overflow na aritmética de ponteiro}
Mensagem de erro: '[exemplo.c:2]: (erro) Overflow na aritmética de ponteiro, ponteiro nulo é subtraído'.
\begin{lstlisting}[style=CStyle]
void foo(char *s) {
    p = s - 20;
}
void bar() { foo(0); }
\end{lstlisting}

\subsubsection{Desreferenciamento de ponteiro nulo}
Mensagem de erro: '[exemplo.c:2]: (erro) Desreferenciamento de ponteiro nulo'.

\begin{lstlisting}[style=CStyle]
int foo(int *p) {
    *p = 0;
    if (!p) {
        return 1;
    }
    return 0;
}
\end{lstlisting}


\subsection{Verificação de Arquivo}

Os erros apontados de verificação de arquivo podem garantir a manipulação correta de um arquivo, como por exemplo, o objetivo do arquivo é de escrita e leitura e o aluno abriu para leitura, portanto não irá conseguir fazer nenhuma escrita no arquivo.

\subsubsection{Operação de escrita em um arquivo que foi aberto somente para leitura}
Mensagem de erro: '[exemplo.c:5]: (erro) Operação de escrita em um arquivo que foi aberto somente para leitura'.
\begin{lstlisting}[style=CStyle]
void foo(FILE*& f) {
    f = fopen(name, "r");
    fread(buffer, 5, 6, f);
    rewind(f);
    fwrite(buffer, 5, 6, f);
}
\end{lstlisting}

\subsubsection{Operação de leitura em um arquivo que foi aberto somente para escrita}
Mensagem de erro: '[exemplo.c:5]: (erro) Operação de leitura em um arquivo que foi aberto somente para escrita'.
\begin{lstlisting}[style=CStyle]
void foo(FILE*& f) {
    f = fopen(name, "w");
    fwrite(buffer, 5, 6, f);
    rewind(f);
    fread(buffer, 5, 6, f);
}
\end{lstlisting}

\subsubsection{Arquivo utilizado que não foi aberto}
Mensagem de erro: '[exemplo.c:3]: (erro) Arquivo utilizado que não foi aberto'.
\begin{lstlisting}[style=CStyle]
void foo() {
    FILE* f;
    fwrite(buffer, 5, 6, f);
}
\end{lstlisting}

\subsubsection{Chamada de função fflush() no stream de entrada}
Mensagem de erro: '[exemplo.c:2]: (erro) Chamada de função fflush() no stream de entrada, podendo resultar em comportamento indefinido em sistemas não-linux'.
\begin{lstlisting}[style=CStyle]
void foo() {
    fflush(stdin);
}
\end{lstlisting}


\subsection{Nenhum erro foi encontrado}
Mensagem: '[exemplo.c:2]: Nenhum erro de análise estática foi encontrado'.

Esta mensagem significa que nenhum dos erros apresentados neste documento foram encontrados no arquivo \textit{exemplo.c}.
Contudo, pode ainda haver erros no código mas que não são cobertos pelo SuSy-avalia.

\printbibliography

\clearpage

\end{document}